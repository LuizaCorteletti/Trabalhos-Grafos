GRUPO:
    Gabriel Martins Rajão - 801792
    Luiza Dias Corteletti - 816706

listaEncadeada:

    Criamos uma estrutura de No composta por um inteiro e um ponteiro para outro No. Adicionamos 2 funções de inserir ( uma no inicio e outra no fim), sempre quando um novo
    elemento é adicionado, é adicionada uma referencia a ele em um dos ponteiros da lista de Nos. Temos também 2 funções de remover ( uma no inicio e outra no fim), sempre quando
    um elemento é removido, a referencia dele nos ponteiros da lista é substituida por um NULL ou outra referencia. Por fim, adicionamos uma funcao de busca que percorre a lista
    procurando um valor e uma funcao de print que percorre a lista imprimindo os valores.

Fila:

    Reutilizamos o codigo da listaEncadeada, porém, a insercao sempre ocorre no inicio e a remocao sempre no final. Dessa forma, o primeiro a chegar é sempre o primeiro a sair.

Pilha:

    Reutilizamos o codigo da listaEncadeada, porém, a insercao sempre ocorre no inicio e a remocao sempre no inicio. Dessa forma, o primeiro a chegar é sempre o ultimo a sair.

Matriz:

    Nós desenvolvemos esse código para manipular uma matriz utilizando listas encadeadas. O código permite a criação de uma matriz de dimensões especificadas pelo usuário e 
    possibilita realizar operações como inserção, remoção, impressão e busca de elementos dentro dessa matriz. 
    A matriz é representada por uma lista encadeada de colunas, onde cada coluna é composta por células que armazenam os elementos. 
    O usuário pode interagir com a matriz inserindo valores em posições específicas, removendo elementos, imprimindo a matriz atual e buscando a posição de um determinado valor 
    na matriz. Além disso, o código inclui tratamento de exceções para lidar com entradas inválidas ou operações em variáveis vazias.