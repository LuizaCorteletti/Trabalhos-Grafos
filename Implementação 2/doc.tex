GRUPO:
    Gabriel Martins Rajão - 801792
    Luiza Dias Corteletti - 816706

# matriz.h #

    Criamos essa classe para construir as representações dos grafos onde as linhas representam 
    os vértices e as colunas as arestas. 

# grafo.h #

    classe Grafo:

    Criamos essa classe Grafo para representar e manipular grafos de maneira eficiente. A classe
    possui o método fazGraComp que  configura o grafo como completo, ou seja, todas as arestas 
    entre os vértices são definidas como 1, exceto as diagonais (loops), que são definidas 
    como 0.

    GeraSubgrafos:

    Implementamos essa função, para gerar e imprimir todos os subgrafos possíveis de um grafo 
    completo com um número específico de vértices. A função faz isso gerando todas as 
    combinações possíveis de arestas, representadas como strings de "0" e "1", onde "0" indica 
    a ausência de uma aresta e "1" indica sua presença.
    Exemplo: Caso a função receba 2 vértices como parâmetro é calculado o número de vértices
    (2*1/2 = 1 aresta). Para essa 1 aresta podem ser gerados os valores 1 e 0:
    1: 
        1: [0,1]
        2: [1,0]
    0: 
        1: [0,0]
        2: [0,0]


    allOpt:

    Desenvolvemos essa função para gerar todas as combinações possiveis entre os vértices de um
    grafo completo e em seguida usar essas combinações para gerar os subgrafos usando a função
    anterior (GeraSubgrafos).
    
    labelGen:

    Função que recebe a quantidade de vértices do grafo e chama a função allOpt para calcular 
    todos os subgrafos possíveis. 

# main.cpp #

    fat:

    Função que calcula o fatorial de um número.

    subgCount:

    Criamos essa função para calcular o número total de subgrafos de um grafo completo 
    utilizando a equação passada em sala. 

    main:

    Contruimos um menu simples para que o usuário digite o número de vértices do grafo a ter 
    seus subgrafos gerados e depois receba o número de subgrafos possíveis e uma representação
    visual para cada um deles. 